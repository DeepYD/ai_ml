\documentclass{beamer}
\usepackage{graphicx}
\usetheme{Warsaw}
\title[]{EE1390}
\subtitle{Matrix Project}
\author{EE18BTECH11027 and EE18BTECH11011}
\date{}

\begin{document}


\begin{frame}
\titlepage
\end{frame}

\begin{frame}{Question}
Let A $\binom{2}{-3}$ and B $\binom{-2}{1}$ be vertices of a triangle ABC. If the centroid of this moves on the line 2x+3y=1, then the locus of the vertex  C is -

\setlength{\parindent}{4cm}
JEE Main - 2004
\end{frame}

\begin{frame}
{\includegraphics[scale=0.8]{../../Downloads/Given.png} 3}
\end{frame}

\begin{frame}{solution}
let A $\binom{2}{-3}$ and B $\binom{-2}{1}$ and C be vertices of the triangle.

and  \\ O be the centroid of the triangle which lies on the line 2x+3y=1 \\ 
  hence, \\ the matrix equation of given line is
  \setlength{\parindent}{4cm}
  \\(2  3)$\boldsymbol{x}$=1
\end{frame}

\begin{frame}
we know that the coordinate of the centroid is the arithmetic mean of the all the three coordinates. \\ therefore,

O = (A+B+C/3)
\\also, \\O lies on the line (2 3)$\boldsymbol{x}$=1
\\hence it satisfies the equation, therfore\\
(2   3)O=1

\end{frame}

\begin{frame}
substituting O = (A+B+C)/3  in - \\
\begin{equation}
 (2\ 3)O=1
\end{equation}
\begin{equation}
(2 \ 3)(A+B+C)/3 = 1
\end{equation}
 
\begin{equation}
(2 \  3)(A+B+C) = 3
\end{equation}
 
\begin{equation}
(2 \  3)C + (2 \ 3)(A+B) = 3 
\end{equation}
\begin{equation}
(2  \ 3)C = 3 - (2\ 3)(A+B)
\end{equation}
\end{frame}


\begin{frame}
 

(2\ 3)C = 3 - (2\ 3) ($\binom{2}{-3}$  +  $\binom{-2}{1}$)  

(2\ 3)C = 3 - (2\ 3) $\binom{0}{-2}$

(2\ 3)C = 3 - (-6)

(2\ 3)C = 9\\



hence the locus of C is
$n^T$
 $\boldsymbol{x}$= 9
where n = $\binom{2}{3}$

\end{frame}

\begin{frame}

\includegraphics[scale=0.8]{../../Downloads/Figure2.png} 
\end{frame}

\end{document} 

  
  